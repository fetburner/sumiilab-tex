\documentclass[cjk,xcolor=dvipsnames,fleqn]{beamer}
\graphicspath{{./figures/}}
\hypersetup{unicode=true}% しおりの文字化け対策
\usepackage[no-math]{fontspec}
\usepackage{luatexja,luatexja-fontspec}
\usepackage{graphicx}
\usepackage{calc}%% Enable infix arithmetic operators

\usepackage[orientation=portrait,size=a0]{beamerposter}
\usetheme{sumiilab-poster}

%% フォントの埋め込み (MigMix 1P http://mix-mplus-ipa.osdn.jp/migmix/)
\setsansfont[% 英語 Sans フォントの指定
  Path=./fonts/migmix-1p-20150712/,
  Extension=.ttf,
  Ligatures=TeX,
  BoldFont=migmix-1p-bold
]{migmix-1p-regular}
\setsansjfont[% 日本語 Sans フォントの指定
  Path=./fonts/migmix-1p-20150712/,
  Extension=.ttf,
  Ligatures=TeX,
  BoldFont=migmix-1p-bold
]{migmix-1p-regular}

\title[Beamer Poster]{住井・松田研究室のステキなポスター}
\author[Lambda \& Pi]{ラムダ小太郎 \and パイ三郎}
\institute[東北大学]{東北大学 大学院情報科学研究科}

\begin{document}
\begin{frame}[t,fragile]{}
  \begin{block}{はじめに}
    \alert{LaTeX} で学会ポスターを作りたい人のために!
    \begin{itemize}
    \item (宗教上の理由で)PowerPoint を使えない人に有用
    \item Beamer でポスターが書ける (beamerposter.sty)
    \item よだれが出るほど美しい数式
    \item \alert{可読性}・\alert{視認性}を重視
      \begin{itemize}
      \item センス極まったデザインにしたい人は勝手にどうぞ
      \end{itemize}
    \end{itemize}
  \end{block}
  \vskip1ex
  \begin{columns}[onlytextwidth]
    \begin{column}[t]{0.5\textwidth-1em}
      \begin{block}{簡単な機能紹介}
        \begin{itemize}
        \item セクション:block 環境を使う
          \begin{itemize}
          \item alertblock, exampleblock も可
          \end{itemize}
        \item 段組:columns, column 環境を使う
        \item フォントサイズ:
          \begin{itemize}
          \item tiny: {\tiny abc あいうえお}
          \item scriptsize: {\scriptsize abc あいうえお}
          \item footnotesize: {\footnotesize abc あいうえお}
          \item small: {\small abc あいうえお}
          \item normalsize: {\normalsize abc あいうえお}
          \item large: {\large abc あいうえお}
          \item Large: {\Large abc あいうえお}
          \item LARGE: {\Large abc あいうえお}
          \item huge: {\huge abc あいうえお}
          \item Huge: {\Huge abc あいうえお}
          \item HUGE: {\HUGE abc あいうえお}
          \end{itemize}
        \item その他、アニメーション以外の Beamer の主な機能はだいたい使える
        \end{itemize}
      \end{block}
      \vskip1ex
    \end{column}
    \begin{column}[t]{0.5\textwidth}
      \begin{alertblock}{節タイトル}
        これは alertblock
      \end{alertblock}
      \begin{exampleblock}{節タイトル}
        これは exampleblock
      \end{exampleblock}
    \end{column}
  \end{columns}
\end{frame}
\end{document}
