\documentclass{sumiilab-paper}
%% uplatex を使う場合:
% \documentclass[uplatex]{sumiilab-paper}

% 論文の年度と種類
\paper{平成 n 年度 卒業論文}% 学部生
%\paper{平成 n 年度 修士論文}% 修士

% 論文のタイトル
\title{住井研究室の\\ステキな論文クラスファイルの使用例}

% 学籍番号と著者のお名前
\author{X0XX1234 ラムダ 小太郎}

% 著者の所属
\institute{東北大学 工学部\\情報知能システム総合学科}% 学部生
%\institute{東北大学 大学院 情報科学研究科\\情報基礎科学専攻}% 修士

% 指導教員のお名前
\supervisor{亀山 充隆 教授}% 指導教員
\subsupervisor{住井 英二郎 准教授}% 論文指導教員(省略可)

% 日付
\date{\today}

%% ===============================================
%% ソースコードの設定
%% ===============================================

\usepackage{listings,jlisting}
\usepackage{DejaVuSansMono}

% プログラミング言語と表示するフォント等の設定
\lstset{
  language={[Objective]Caml},% プログラミング言語
  basicstyle={\ttfamily\small},% ソースコードのテキストのスタイル
  keywordstyle={\bfseries},% 予約語等のキーワードのスタイル
  commentstyle={},% コメントのスタイル
  stringstyle={},% 文字列のスタイル
  frame=trlb,% ソースコードの枠線の設定 (none だと非表示)
  numbers=left,% 行番号の表示 (none だと非表示)
  numberstyle={\footnotesize},% 行番号のスタイル
  xleftmargin=15pt,% 左余白
  xrightmargin=5pt,% 右余白
  keepspaces=true,% 空白を維持する
  mathescape,% $ で囲った部分を数式として表示する ($ がソースコード中で使えなくなるので注意)
  % 手動強調表示の設定
  moredelim=[is][\bfseries]{@*}{*@},
  moredelim=[is][\itshape]{@/}{/@}
}

\begin{document}
\maketitle

\begin{abstract}
ステキな論文の概要
\end{abstract}

\chapter*{謝辞}

ステキな論文の謝辞

%% 目次
\tableofcontents

%% ここから本文

\chapter{序論}

%% 参考文献は \cite{ID} とします(ID は refs.bib 内で文献につけた識別子)
%% BibTeX の使い方などは各自調べて下さい。
序論とか本論とか結論とか \cite{TAPL}

\chapter{本論}

ソースコード\ref{src:listup_nodes}は二分木を深さ優先探索して、ノードを列挙する関数である。
\begin{lstlisting}[caption=二分木のノードのリストアップ,label=src:listup_nodes]
type 'a bin_tree =
  | Leaf of 'a
  | Node of 'a bin_tree * 'a bin_tree

let rec listup_nodes = function
  | Leaf x -> [x]
  | Node (r, l) -> (listup_nodes r) @ (listup_nodes l)
\end{lstlisting}
ソースコードの書き方等については slide ブランチの slide.tex を参照されたし。

\chapter{結論}

%% 参考文献: bibtex (\cite が1つも無いとコンパイルエラーになるので注意)
\bibliographystyle{junsrt}
\bibliography{refs}

\end{document}
